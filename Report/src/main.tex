\documentclass[12pt]{report}
\usepackage{arabicore}
\input{setup}

\begin{document}

    \begin{titlepage}
        \centering

        \renewcommand{\arraystretch}{1.1} % Increase row height
        \begin{tabularx}{\textwidth}{@{}m{0.9\textwidth}X@{}}
            \centering \raggedleft \cellcolor{lightgray!25} Αλέξανδρος Ξιάρχος & \centering\cellcolor{darkgray!40}\fontDin \raisebox{-0.2ex}{1059619}
        \end{tabularx}

        \vspace*{10em}
        \begin{headerlight}
            \begin{Din}
                \centering
                    {ΠΑΝΕΠΙΣΤΗΜΙΟ ΠΑΤΡΩΝ \(\cdot\) ΤΜΗΜΑ ΜΗΧΑΝΙΚΩΝ Η/Υ ΚΑΙ ΠΛΗΡΟΦΟΡΙΚΗΣ}
            \end{Din}
        \end{headerlight}

        \begin{headerdark}
            \begin{Din Medium}
                \centering
                \huge \textcolor{white}{ΑΝΑΚΤΗΣΗ ΠΛΗΡΟΦΟΡΙΑΣ}
            \end{Din Medium}
        \end{headerdark}

        \begin{headerlight}
            \begin{Din}
                \centering
                    ΕΡΓΑΣΤΗΡΙΑΚΗ ΑΣΚΗΣΗ \(\cdot\)2023-2024
            \end{Din}
        \end{headerlight}

    \end{titlepage}


    \tableofcontents
    \pagebreak


    \chapter{ΕΙΣΑΓΩΓΗ}

         \section{ΥΠΟΛΟΓΙΣΜΟΣ ΒΑΡΩΝ TF \& IDF}

            Καταρχάς πρέπει να επιλέξουμε την παραλλαγή των βαρών TF και IDF για έγγραφα και ερωτήματα που είναι καταλληλότερη για τη συλλογή μας.

            Όσον αφορά τα \textbf{έγγραφα}: Μιας και η συλλογή αφορά βάση δεδομένων για τηn Κυστική Ίνωση, δηλαδή πρόκειται για συλλογή
            με τεχνικές –ιατρικές συγκεκριμένα– ορολογίες {\fontTimes(technical vocabulary and meaningful terms [MED collections])}
                \footnote{Gerard Salton, Christopher Buckley, Term-weighting approaches in automatic text retrieval, Information Processing \& Management, Volume 24, Issue 5, 1988, Pages 513-523, ISSN 0306-4573},
            θα χρησιμοποιήσουμε τη \textbf{διπλή 0,5 κανονικοποίηση} {\fontTimes(augmented normalized TF)}: \[ 0.5 + 0.5 \frac{F_{ij}}{max_k F_{kj}} \]
            για το βάρος που αφορά τα έγγραφα, όπου \boldmath{\(F_{ij}\)} οι φορές που ο όρος εμφανίζεται σε ένα έγγραφο
                και \boldmath{\( \max_k{F_{kj}}\)} το μεγαλύτερο πλήθος εμφανίσεων κάποιου όρου σε ένα έγγραφο.

            Όσον αφορά τα \textbf{queries}: κάθε λήμμα από τα ερωτήματα είναι σημαντικό (σχεδόν κάθε λέξη είναι ιατρική ορολογία),
            άρα θα χρησιμοποιήσουμε πάλι τη \textbf{διπλή 0,5 κανονικοποίηση} για το TF βάρος.

            Για το IDF βάρος και σε έγγραφα και σε ερωτήματα, χρησιμοποιούμε την \textbf{απλή ανάστροφη συχνότητα εμφάνισης}:
               \[\log{\frac{N}{n_i}} \] όπου \boldmath{\(N\)} ο συνολικός αριθμός των εγγράφων και \boldmath{\(n_i\)} ο αριθμός των εγγράφων στα οποία εμπεριέχεται ο όρος.

            \noindent
            \begin{tblr}{
                colspec={>{\centering\arraybackslash}m{3.5cm}>{\centering\arraybackslash}m{6.1cm}>{\centering\arraybackslash}m{6.1cm}},
                row{2}={bg=lightgray!50}, row{3}={bg=lightgray}, row{1}={bg=black!90,fg=white}}
                 \bgcolor{white} & Βάρος εγγράφων &  Βάρος ερωτημάτων \\
                 \textbf{TF} & \[ 0.5 + 0.5 \frac{F_{ij}}{max_k F_{kj}} \] & \[ 0.5 + 0.5 \frac{F_{ij}}{max_k F_{kj}} \] \\
                 \textbf{IDF} & \[\log{\frac{N}{n_i}} \] & \[\log{\frac{N}{n_i}} \] \\
            \end{tblr}

        \pagebreak
    \chapter{ΥΛΟΠΟΙΗΣΗ}
        Η υλοποίηση έχει χωριστεί στα εξής αρχεία: \\


        \section{ΠΡΟΕΠΕΞΕΡΓΑΣΙΑ ΕΓΓΡΑΦΩΝ \& ΒΟΗΘΗΤΙΚΕΣ ΣΥΝΑΡΤΗΣΕΙΣ}

            Το αρχείο {\fontCode\small tools.py} περιλαμβάνει βοηθητικές συναρτήσεις για κάποιες επαναλαμβανόμενες διαδικασίες της υλοποίησης.
            Περιλαμβάνονται οι συναρτήσεις {\fontCode\small \textbf{get\_docs}()} και {\fontCode\small \textbf{get\_queries}()}.

            Η συνάρτηση {\fontCode\small \textbf{get\_docs}()}, χρησιμοποιώντας την {\fontCode\small os} βιβλιοθήκη
            διαβάζει το πλήθος των αρχείων της βιβλιοθήκης.\footnote{Να σημειωθεί ότι το πλήθος των εγγράφων διαφέρει από την αύξουσα αρίθμησή τους. Συγκεκριμένα έχουμε 1209 έγγραφα αριθμημένα από το {\fontCode\scriptsize 000001} ως {\fontCode\scriptsize 01239}. Με άλλα λόγια υπάρχουν αριθμοί στη συλλογή που δεν αντιστοιχούν σε έγγραφα. Συνεπώς δεν θα μπορούσαμε να χρησιμοποιήσουμε κάποια αριθμητική επανάληψη, για παράδειγμα, για την εισαγωγή των εγγράφων.}
            Η συνάρτηση δημιουργεί και επιστρέφει μια λίστα από tuples, με κάθε tuple να αντιστοιχεί σε κάθε αρχείο-έγγραφο. Τα tuples έχουν την δομή:

                \begin{graycomment} \centering
                {\fontCode\footnotesize ('docID', ['λήμμα\_1', 'λήμμα\_2' ...])}
                \end{graycomment}

            \noindent όπου {\fontCode\small docID} η αρίθμηση του κάθε εγγράφου και {\fontCode\small doc\_term\_n} η κάθε λέξη-λήμμα του εγγράφου.
            Η συνάρτηση {\fontCode\small strip()} είναι απαραίτητη για την αφαίρεση των {\fontCode\small \textbackslash n} χαρακτήρων που προέκυψαν από την μορφολογία των εγγράφων (κάθε λέξη είναι σε νέα γραμμή).
            Αντίστοιχα η συνάρτηση {\fontCode\small \textbf{get\_queries}()} επιστρέφει ????????????

            Στις συναρτήσεις {\fontCode\small \textbf{preprocess\_collection}()} και {\fontCode\small \textbf{preprocess\_queries}()} πραγματοποιείται η προεπεξεργασία των εγγράφων, συγκεκριμένη η αφαίρεση των \textbf{stopwords} και το \textbf{stemming}.


            Η αφαίρεση των stopwords και το stemming γίνεται με τη χρήση της {\fontCode\small nltk} βιβλιοθήκης. Τα {\fontCode\small doc\_tuples} της {\fontCode\small get\_docs()} αφού περάσουν από τον {\fontCode\small PorterStemmer} της {\fontCode\small nltk} αποθηκεύονται σε μια λίστα, η οποία στη συνέχεια αποθηκεύεται ως ένα {\fontCode\small .json} αρχείο.
            Αντίστοιχη διαδικασία πραγματοποιείται και για την προεπεξεργασία των ερωτημάτων, στην {\fontCode\small preprocess\_queries()}

        \section{ΑΝΕΣΤΡΑΜΜΕΝΟ ΕΥΡΕΤΗΡΙΟ}

            Το ανεστραμμένο ευρετήριο δημιουργείται στη συνάρτηση {\fontCode\small \textbf{create\_inverted\_index}()} του αρχείου {\fontCode\small inverted\_index.py}.
            Στην συνάρτηση εισαγάγονται τα {\fontCode\small .json} αρχεία που δημιουργήθηκαν στις προηγούμενες συναρτήσεις.

            Τα tuples που αντιστοιχούν σε αυτά αποθηκεύονται σε ένα dictionary που θα αποτελέσει το ανεστραμμένο ευρετήριο με την εξής δομή:

                \begin{graycomment} \centering
                    {\fontCode\footnotesize inverted\_index['λήμμα'] = \\ \{ ('docID στο οποίο εμφανίζεται' = <φορές εμφάνισης>), (\(\cdots\)), \(\ldots\) \}}
                \end{graycomment}

            Κάθε value του dictionary είναι ένα set\footnote{Έχει επιλεχθεί set για εξοικονόμιση μνήμης, μας και δεν μας ενδιαφέρει η σειρά των tuples.} το οποίο περιλαμβάνει ένα ή περισσότερα tuples με το {\fontCode\small docID} και τη συχνότητα εμφάνισης του λήμματος στο συγκεκριμένο έγγραφο.
            Η συχνότητα υπολογίζεται μέσω της {\fontCode\small count()} σε όλο το έγγραφο ανά λήμμα. Αυτό είναι ένα παράδειγμα του τελικού ανεστραμμένου ευρετηρίου:


                \begin{graycomment} \centering
                    {\fontCode\scriptsize inverted\_index = \{\(\ldots\) 'coronari': {('01217', 2), ('00779', 1), ('00164', 1)}, \\ 'graft': {('00164', 1)}, 'mobil': {('00673', 2), 'strain': {('00179', 7), \(\ldots\)\} }
                \end{graycomment}

        \section{ΥΛΟΠΟΙΗΣΗ VECTOR SPACE ΜΟΝΤΕΛΟΥ}



\end{document}