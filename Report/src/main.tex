\documentclass[12pt]{report}
\usepackage{arabicore}
\usepackage{geometry}
\usepackage{fontspec}
\usepackage{xcolor}
\usepackage{colortbl}
\usepackage{tabularx}
\usepackage{tabularray}
\UseTblrLibrary{booktabs}
\usepackage{amstex}
\usepackage{amsmath}
\usepackage{tcolorbox}
\usepackage{titlesec}
\usepackage{setspace}
\usepackage[greek]{babel}
\usepackage{indentfirst}
\usepackage{fancyhdr}
%\usepackage[fontsize=13pt]{fontsize}

\geometry{a4paper, total={170mm,260mm}, left=20mm, top=20mm}

\setmainfont{Conduit ITC Hel Light}
\newfontfamily\fontDin{CF Din Condensed}
\newenvironment{Din}{\fontDin}{\par}
\newfontfamily\fontDinLight{CF Din Light Condensed}
\newenvironment{Din Light}{\fontDinLight}{\par}
\newfontfamily\fontDinMedium{CF Din Medium Condensed}
\newenvironment{Din Medium}{\fontDinMedium}{\par}
\newfontfamily\fontCode{Courier New}
\newenvironment{Code}{\fontCode}{\par}
\newfontfamily\fontTimes{Times New Roman}
\newenvironment{Times}{\fontTimes}{\par}

\newfontfamily\headingfont[]{CF Din Medium Condensed}

\addto\captionsgreek{% Replace "english" with the language you use
  \renewcommand{\contentsname}%
    {ΠΕΡΙΕΧΟΜΕΝΑ}%
}

\newtcolorbox{headerdark}{colback=darkgray, boxrule=0pt,arc=0pt, boxsep=12pt,left=2pt,right=2pt,leftrule=0pt}
\newtcolorbox{headerlight}{colback=gray!50, boxrule=0pt,arc=0pt, boxsep=2pt,left=2pt,right=2pt,leftrule=0pt}
\newtcolorbox{graycomment}{colback=gray!25, boxrule=0pt,arc=0pt, boxsep=2pt,left=2pt,right=2pt,leftrule=1pt, grow to left by=-28pt,  grow to right by=-28pt}

% Το στιλ των νέων κεφαλαίων/sectionσ
\titleformat{\chapter}[block]
    {\color{black} \fontDinLight \large } % global formatting (number and title)
    {\color{white} \fontDin \colorbox{black!80} {\hspace{7pt}\thechapter\hspace{7pt}}} % label: number and its formatting
    {} % spacing between number and title
    {\colorbox{gray!25}} % optional (content between number and title)

\titleformat{\section}[block]
    {\color{black} \fontDinLight \large} % global formatting (number and title)
    {\color{white} \fontDin \colorbox{black!80} {\hspace{5pt}\thesection\hspace{5pt}}} % label: number and its formatting
    {} % spacing between number and title
    {\colorbox{gray!25}} % optional (content between number and title)

\titlespacing*{\chapter}
   {0pt}{0}{0.4em}  % left before after

\titlespacing*{\section}
   {0pt}{1em}{0.4em}  % left before after

\titleclass{\chapter}{straight}


%% Setting for the column separator
%\colorlet{shadecol}{black!20}
%\setlength\columnsep{12pt}
%\makeatletter   % This change vertical bar to a dotted line
%\newcommand{\latexcolumnseprulecolor}{\color{shadecol}}
%\renewcommand\dotfill[1][0.4em]{%
%  \leavevmode\cleaders\hb@xt@ #1{\hss .\hss}\hfill\kern\z@}
%\patchcmd{\@outputdblcol}%
%  {\vrule\@width\columnseprule}%
%  {\rotatebox{90}{\parbox{\textheight}{\dotfill[0.3em]}}}%
%  {}{}
%\makeatother
%
%% Command to render shaded heading
%\newsavebox\labbox
%\NewDocumentCommand\shadedsec{O{1.5\baselineskip} O{0.33\dimexpr#1} m}{%
%  \sbox\labbox{%
%    \colorbox{black!80}{%
%      \textcolor{white}{\hspace{0.5em}#3\hspace{0.5em}}}}
%  \addvspace{#1}
%  \noindent%
%  \nopagebreak%
%  \usebox\labbox
%  {\color{shadecol}\rule[-\fboxsep]{\dimexpr\columnwidth-\wd\labbox}{\dimexpr\ht\labbox+\fboxsep}}%
%  \vspace{#2}\par}


% Ρύθμση αλλαγής γραμμής
\tolerance=1
\emergencystretch=\maxdimen
\hyphenpenalty=10000 % Για να μην κάνει συλλαβισμό στις λέξεις
\hbadness=10000

% Αλλαγή απόστασης μεταξύ παραγράφων
\setlength{\parskip}{6pt}

% Ρύθμιση Headers/Footers
\pagestyle{fancy}
\renewcommand{\headrulewidth}{0pt}
\fancyhead{}\fancyfoot{}
\fancyhead[R]{\fontDinLight ΑΛΕΞΑΝΔΡΟΣ ΞΙΑΡΧΟΣ \(\cdot\) 1059619\hspace{10pt}\colorbox{black!50}{\color{white}\fontDin\thepage}}

% Συνεχόμενη αρίθμηση ανά chapters
\counterwithout{footnote}{chapter}

\endinput


\begin{document}

    \begin{titlepage}
        \centering

        \renewcommand{\arraystretch}{1.1} % Increase row height
        \begin{tabularx}{\textwidth}{@{}m{0.9\textwidth}X@{}}
            \centering \raggedleft \cellcolor{lightgray!25} Αλέξανδρος Ξιάρχος & \centering\cellcolor{darkgray!40}\fontDin \raisebox{-0.2ex}{1059619}
        \end{tabularx}

        \vspace*{10em}
        \begin{headerlight}
            \begin{Din}
                \centering
                    {ΠΑΝΕΠΙΣΤΗΜΙΟ ΠΑΤΡΩΝ \(\cdot\) ΤΜΗΜΑ ΜΗΧΑΝΙΚΩΝ Η/Υ ΚΑΙ ΠΛΗΡΟΦΟΡΙΚΗΣ}
            \end{Din}
        \end{headerlight}

        \begin{headerdark}
            \begin{Din Medium}
                \centering
                \huge \textcolor{white}{ΑΝΑΚΤΗΣΗ ΠΛΗΡΟΦΟΡΙΑΣ}
            \end{Din Medium}
        \end{headerdark}

        \begin{headerlight}
            \begin{Din}
                \centering
                    ΕΡΓΑΣΤΗΡΙΑΚΗ ΑΣΚΗΣΗ \(\cdot\)2023-2024
            \end{Din}
        \end{headerlight}

    \end{titlepage}


    \tableofcontents
    \pagebreak


    \chapter{ΕΙΣΑΓΩΓΗ}

    \section{VECTOR SPACE MODEL}
        % SAMPLE ΚΕΙΜΕΝΟ
        Το μοντέλο διανυσματικού χώρου (Vector Space Model) αποτελεί μία από τις πιο διαδεδομένες μεθόδους για την αναπαράσταση και επεξεργασία κειμένων
        σε συστήματα ανάκτησης πληροφορίας. Σύμφωνα με αυτό, κάθε κείμενο αναπαρίσταται ως ένα διάνυσμα σε ένα πολυδιάστατο χώρο, όπου κάθε διάσταση αντιστοιχεί
        σε ένα όρο του λεξιλογίου. Το μήκος του διανύσματος αντιστοιχεί στη σημασία του κειμένου, ενώ η κατεύθυνση του διανύσματος αντιστοιχεί στο περιεχόμενο του κειμένου.
        Η ομοιότητα μεταξύ δύο κειμένων μπορεί να υπολογιστεί ως η συνημιτονική γωνία μεταξύ των αντίστοιχων διανυσμάτων. Με αυτόν τον τρόπο,
        μπορούμε να ταξινομήσουμε τα έγγραφα μιας συλλογής ως προς την ομοιότητά τους με ένα δεδομένο ερώτημα.


    \section{ΥΠΟΛΟΓΙΣΜΟΣ ΒΑΡΩΝ TF \& IDF}
    \par Καταρχάς πρέπει να επιλέξουμε την παραλλαγή των βαρών TF και IDF για έγγραφα και ερωτήματα που είναι καταλληλότερη για τη συλλογή μας.

    \par Όσον αφορά τα \textbf{έγγραφα}: Μιας και η συλλογή αφορά βάση δεδομένων για τηn Κυστική Ίνωση, δηλαδή πρόκειται για συλλογή
    με τεχνικές –ιατρικές συγκεκριμένα– ορολογίες {\fontTimes(technical vocabulary and meaningful terms [MED collections])}
        \footnote{Gerard Salton, Christopher Buckley, Term-weighting approaches in automatic text retrieval, Information Processing & Management, Volume 24, Issue 5, 1988, Pages 513-523, ISSN 0306-4573},
    θα χρησιμοποιήσουμε τη \textbf{διπλή 0,5 κανονικοποίηση} {\fontTimes(augmented normalized TF)}: \[ 0.5 + 0.5 \frac{F_{ij}}{max_k F_{kj}} \]
    για το βάρος που αφορά τα έγγραφα, όπου \boldmath{\(F_{ij}\)} οι φορές που ο όρος εμφανίζεται σε ένα έγγραφο
        και \boldmath{\( \max_k{F_{kj}}\)} το μεγαλύτερο πλήθος εμφανίσεων κάποιου όρου σε ένα έγγραφο.

    \par Όσον αφορά τα \textbf{queries}: κάθε λήμμα από τα ερωτήματα είναι σημαντικό (σχεδόν κάθε λέξη είναι ιατρική ορολογία),
    άρα θα χρησιμοποιήσουμε πάλι τη \textbf{διπλή 0,5 κανονικοποίηση} για το TF βάρος.

    \par Για το IDF βάρος και σε έγγραφα και σε ερωτήματα, χρησιμοποιούμε την \textbf{απλή ανάστροφη συχνότητα εμφάνισης}:
       \[\log{\frac{N}{n_i}} \] όπου \boldmath{\(N\)} ο συνολικός αριθμός των εγγράφων και \boldmath{\(n_i\)} ο αριθμός των εγγράφων στα οποία εμπεριέχεται ο όρος.

    \noindent
    \begin{tblr}{
        colspec={>{\centering\arraybackslash}m{3.5cm}>{\centering\arraybackslash}m{6.1cm}>{\centering\arraybackslash}m{6.1cm}},
        row{2}={bg=lightgray!50}, row{3}={bg=lightgray}, row{1}={bg=black!90,fg=white}}
         \bgcolor{white} & Βάρος εγγράφων &  Βάρος ερωτημάτων \\
         \textbf{TF} & \[ 0.5 + 0.5 \frac{F_{ij}}{max_k F_{kj}} \] & \[ 0.5 + 0.5 \frac{F_{ij}}{max_k F_{kj}} \] \\
         \textbf{IDF} & \[\log{\frac{N}{n_i}} \] & \[\log{\frac{N}{n_i}} \] \\
    \end{tblr}

    \pagebreak
    \chapter{ΥΛΟΠΟΙΗΣΗ}
\end{document}