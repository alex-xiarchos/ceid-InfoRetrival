\documentclass[12pt]{report}
\usepackage{arabicore}
\usepackage{geometry}
\usepackage{fontspec}
\usepackage{xcolor}
\usepackage{colortbl}
\usepackage{tabularx}
\usepackage{tabularray}
\UseTblrLibrary{booktabs}
\usepackage{amstex}
\usepackage{amsmath}
\usepackage{tcolorbox}
\usepackage{titlesec}
\usepackage{setspace}
\usepackage[greek]{babel}
\usepackage{indentfirst}
\usepackage{fancyhdr}
%\usepackage[fontsize=13pt]{fontsize}

\geometry{a4paper, total={170mm,260mm}, left=20mm, top=20mm}

\setmainfont{Conduit ITC Hel Light}
\newfontfamily\fontDin{CF Din Condensed}
\newenvironment{Din}{\fontDin}{\par}
\newfontfamily\fontDinLight{CF Din Light Condensed}
\newenvironment{Din Light}{\fontDinLight}{\par}
\newfontfamily\fontDinMedium{CF Din Medium Condensed}
\newenvironment{Din Medium}{\fontDinMedium}{\par}
\newfontfamily\fontCode{Courier New}
\newenvironment{Code}{\fontCode}{\par}
\newfontfamily\fontTimes{Times New Roman}
\newenvironment{Times}{\fontTimes}{\par}

\newfontfamily\headingfont[]{CF Din Medium Condensed}

\addto\captionsgreek{% Replace "english" with the language you use
  \renewcommand{\contentsname}%
    {ΠΕΡΙΕΧΟΜΕΝΑ}%
}

\newtcolorbox{headerdark}{colback=darkgray, boxrule=0pt,arc=0pt, boxsep=12pt,left=2pt,right=2pt,leftrule=0pt}
\newtcolorbox{headerlight}{colback=gray!50, boxrule=0pt,arc=0pt, boxsep=2pt,left=2pt,right=2pt,leftrule=0pt}
\newtcolorbox{graycomment}{colback=gray!25, boxrule=0pt,arc=0pt, boxsep=2pt,left=2pt,right=2pt,leftrule=1pt, grow to left by=-28pt,  grow to right by=-28pt}

% Το στιλ των νέων κεφαλαίων/sectionσ
\titleformat{\chapter}[block]
    {\color{black} \fontDinLight \large } % global formatting (number and title)
    {\color{white} \fontDin \colorbox{black!80} {\hspace{7pt}\thechapter\hspace{7pt}}} % label: number and its formatting
    {} % spacing between number and title
    {\colorbox{gray!25}} % optional (content between number and title)

\titleformat{\section}[block]
    {\color{black} \fontDinLight \large} % global formatting (number and title)
    {\color{white} \fontDin \colorbox{black!80} {\hspace{5pt}\thesection\hspace{5pt}}} % label: number and its formatting
    {} % spacing between number and title
    {\colorbox{gray!25}} % optional (content between number and title)

\titlespacing*{\chapter}
   {0pt}{0}{0.4em}  % left before after

\titlespacing*{\section}
   {0pt}{1em}{0.4em}  % left before after

\titleclass{\chapter}{straight}


%% Setting for the column separator
%\colorlet{shadecol}{black!20}
%\setlength\columnsep{12pt}
%\makeatletter   % This change vertical bar to a dotted line
%\newcommand{\latexcolumnseprulecolor}{\color{shadecol}}
%\renewcommand\dotfill[1][0.4em]{%
%  \leavevmode\cleaders\hb@xt@ #1{\hss .\hss}\hfill\kern\z@}
%\patchcmd{\@outputdblcol}%
%  {\vrule\@width\columnseprule}%
%  {\rotatebox{90}{\parbox{\textheight}{\dotfill[0.3em]}}}%
%  {}{}
%\makeatother
%
%% Command to render shaded heading
%\newsavebox\labbox
%\NewDocumentCommand\shadedsec{O{1.5\baselineskip} O{0.33\dimexpr#1} m}{%
%  \sbox\labbox{%
%    \colorbox{black!80}{%
%      \textcolor{white}{\hspace{0.5em}#3\hspace{0.5em}}}}
%  \addvspace{#1}
%  \noindent%
%  \nopagebreak%
%  \usebox\labbox
%  {\color{shadecol}\rule[-\fboxsep]{\dimexpr\columnwidth-\wd\labbox}{\dimexpr\ht\labbox+\fboxsep}}%
%  \vspace{#2}\par}


% Ρύθμση αλλαγής γραμμής
\tolerance=1
\emergencystretch=\maxdimen
\hyphenpenalty=10000 % Για να μην κάνει συλλαβισμό στις λέξεις
\hbadness=10000

% Αλλαγή απόστασης μεταξύ παραγράφων
\setlength{\parskip}{6pt}

% Ρύθμιση Headers/Footers
\pagestyle{fancy}
\renewcommand{\headrulewidth}{0pt}
\fancyhead{}\fancyfoot{}
\fancyhead[R]{\fontDinLight ΑΛΕΞΑΝΔΡΟΣ ΞΙΑΡΧΟΣ \(\cdot\) 1059619\hspace{10pt}\colorbox{black!50}{\color{white}\fontDin\thepage}}

% Συνεχόμενη αρίθμηση ανά chapters
\counterwithout{footnote}{chapter}

\endinput


\begin{document}

    \begin{titlepage}
        \centering

        \renewcommand{\arraystretch}{1.1} % Increase row height
        \begin{tabularx}{\textwidth}{@{}m{0.9\textwidth}X@{}}
            \centering \raggedleft \cellcolor{lightgray!25} Αλέξανδρος Ξιάρχος & \centering\cellcolor{darkgray}\fontDin \raisebox{-1pt}{\color{white}1059619}
        \end{tabularx}

        \vspace*{10em}
        \begin{headerlight}
            \begin{Din}
                \centering
                    {ΠΑΝΕΠΙΣΤΗΜΙΟ ΠΑΤΡΩΝ \(\cdot\) ΤΜΗΜΑ ΜΗΧΑΝΙΚΩΝ Η/Υ ΚΑΙ ΠΛΗΡΟΦΟΡΙΚΗΣ}
            \end{Din}
        \end{headerlight}

        \begin{headerdark}
            \begin{Din Medium}
                \centering
                \huge \textcolor{white}{ΑΝΑΚΤΗΣΗ ΠΛΗΡΟΦΟΡΙΑΣ}
            \end{Din Medium}
        \end{headerdark}

        \begin{headerlight}
            \begin{Din}
                \centering
                    ΕΡΓΑΣΤΗΡΙΑΚΗ ΑΣΚΗΣΗ \(\cdot\) 2023 \(\textendash\) 2024
            \end{Din}
        \end{headerlight}

    \end{titlepage}


    \tableofcontents
    \pagebreak


    \chapter{ΕΙΣΑΓΩΓΗ}

        \section{ΤΟ ΔΙΑΝΥΣΜΑΤΙΚΟ ΜΟΝΤΕΛΟ - VECTOR SPACE MODEL}

        \section{ΕΠΙΛΟΓΗ ΣΥΣΤΗΜΑΤΩΝ ΣΤΑΘΜΙΣΗΣ}

        Καταρχάς πρέπει να επιλέξουμε δύο συστήματα στάθμισης των βαρών για τους όρους των εγγράφων και των ερωτημάτων.

            \subsection{ΠΡΩΤΟ ΣΥΣΤΗΜΑ ΣΤΑΘΜΙΣΗΣ}
            Το πρώτο σύστημα στάθμισης θα είναι μια παραλλαγή του προτεινόμενου ως καλύτερου συστήματος σύμφωνα με τους Salton-Buckley\footnote{Gerard Salton, Christopher Buckley, Term-weighting approaches in automatic text retrieval, Information Processing \& Management, Volume 24, Issue 5, 1988, Pages 513-523, ISSN 0306-4573}
            {\fontTimes (best fully weighted system)}. Θα χρησιμοποιήσουμε τη \textbf{απλή συχνότητα εμφάνισης} {\fontTimes (raw term frequency)} για το TF βάρος των εγγράφων:
            \[ \text{\textbf{Σύστημα \#1:}\hspace{5pt}TF βάρος}_{\scriptsize \text{εγγράφων}}=\hspace{5pt} f_{i,j} \]
            όπου \(f_{ij}\) οι φορές που ο όρος εμφανίζεται σε ένα έγγραφο, την \textbf{διπλή 0,5 κανονικοποίηση} για το TF βάρος των ερωτημάτων {\fontTimes(augmented normalized TF)}:
            \[ \text{\textbf{Σύστημα \#1:}\hspace{5pt}TF βάρος}_{\scriptsize \text{ερωτημάτων}}=\hspace{5pt} 0.5 + 0.5 \frac{f_{i,j}}{max_i\hspace{3pt}f_{i,j}} \]
            και τέλος την \textbf{απλή ανάστροφη συχνότητα εμφάνισης} για το IDF βάρος και των εγγράφων και των ερωτημάτων:
            \[\text{\textbf{Σύστημα \#1:}\hspace{5pt}IDF βάρος}_{\scriptsize\begin{matrix}\text{εγγράφων}\\\text{ερωτημάτων}\end{matrix}}=\hspace{5pt} \log{\frac{N}{n_i}} \]
            όπου \(N\) το πλήθος των εγγράφων και \(n_i\) ο αριθμός των εγγράφων στα οποία εμπεριέχεται ο όρος.\footnote{Το σύστημα αναφέρεται ως παραλλαγή των Salton-Buckley για το λόγο ότι δεν έχει συμπεριληφθεί κάποιος παράγοντας κανονικοποίσης, μιας και τα έγγραφα είναι \textit{περίπου} ισομεγέθη (μέσος όρος 350 λέξεις).}


            \subsection{ΔΕΥΤΕΡΟ ΣΥΣΤΗΜΑ ΣΤΑΘΜΙΣΗΣ}

            Στο δεύτερο σύστημα στάθμισης {\fontTimes (best weighted probabilistic weight)} θα χρησιμοποιήσουμε την \textbf{διπλή 0,5 κανονικοποίηση} για το TF βάρος των εγγράφων:
            \[ \text{\textbf{Σύστημα \#2:}\hspace{5pt}TF βάρος}_{\scriptsize \text{εγγράφων}}=\hspace{5pt} 1 + \log f_{i,j} \]
            το \textbf{μοναδιαίο} σύστημα για το IDF βάρος των εγγράφων και την \textbf{απλή λογαριθμική κανονικοποίηση} για το IDF βάρος των ερωτημάτων:
            \[ \text{\textbf{Σύστημα \#2:}\hspace{5pt}IDF βάρος}_{\scriptsize \text{ερωτημάτων}}=\hspace{5pt} \log(1 + \frac{N}{n_i}) \]

            Συνολικά έχουμε τα παρακάτω συστήματα:

        \noindent
        \begin{tblr}{
            colspec={>{\centering\arraybackslash}m{3.5cm}>{\centering\arraybackslash}m{6.1cm}>{\centering\arraybackslash}m{6.1cm}},
            row{2}={bg=lightgray!50}, row{3}={bg=lightgray}, row{1}={bg=black!90,fg=white}}
            Σύστημα στάθμισης & Βάρος όρου εγγράφου &  Βάρος όρου ερωτήματος \\
            1 & \[f_{i,j} \times \log{\frac{N}{n_i}} \] & \[(0.5 + 0.5 \frac{f_{i,j}}{max_i\hspace{3pt}f_{i,j}}) \times \log{\frac{N}{n_i}} \] \\
            2 & \[1 + \log f_{i,j}\] & \[ \log(1 + \frac{N}{n_i}) \] \\
        \end{tblr}
        \\\\


    \chapter{ΥΛΟΠΟΙΗΣΗ}
        Η υλοποίηση έχει χωριστεί στα εξής αρχεία: \\

        \section{ΠΡΟΕΠΕΞΕΡΓΑΣΙΑ ΕΓΓΡΑΦΩΝ \& ΒΟΗΘΗΤΙΚΕΣ ΣΥΝΑΡΤΗΣΕΙΣ}

            Το αρχείο {\fontCode\small tools.py} περιλαμβάνει βοηθητικές συναρτήσεις για κάποιες επαναλαμβανόμενες διαδικασίες της υλοποίησης.
            Περιλαμβάνονται οι συναρτήσεις {\fontCode\small \textbf{get\_docs}()} και {\fontCode\small \textbf{get\_queries}()}.

            Η συνάρτηση {\fontCode\small \textbf{get\_docs}()}, χρησιμοποιώντας την {\fontCode\small os} βιβλιοθήκη
            διαβάζει το πλήθος των αρχείων της βιβλιοθήκης.\footnote{Να σημειωθεί ότι το πλήθος των εγγράφων διαφέρει από την αύξουσα αρίθμησή τους. Συγκεκριμένα έχουμε 1209 έγγραφα αριθμημένα από το {\fontCode\scriptsize 000001} ως {\fontCode\scriptsize 01239}. Με άλλα λόγια υπάρχουν αριθμοί στη συλλογή που δεν αντιστοιχούν σε έγγραφα. Συνεπώς δεν θα μπορούσαμε να χρησιμοποιήσουμε κάποια αριθμητική επανάληψη, για παράδειγμα, για την εισαγωγή των εγγράφων.}
            Η συνάρτηση δημιουργεί και επιστρέφει μια λίστα από tuples, με κάθε tuple να αντιστοιχεί σε κάθε αρχείο-έγγραφο. Τα tuples έχουν την δομή:

                \begin{graycomment} \centering
                {\fontCode\footnotesize ('docID', ['λήμμα\_1', 'λήμμα\_2' ...])}
                \end{graycomment}

            \noindent όπου {\fontCode\small docID} η αρίθμηση του κάθε εγγράφου και {\fontCode\small doc\_term\_n} η κάθε λέξη-λήμμα του εγγράφου.
            Η συνάρτηση {\fontCode\small strip()} είναι απαραίτητη για την αφαίρεση των {\fontCode\small \textbackslash n} χαρακτήρων
            που προέκυψαν από την μορφολογία των εγγράφων (κάθε λέξη είναι σε νέα γραμμή).
            Αντίστοιχα η συνάρτηση {\fontCode\small \textbf{get\_queries}()} επιστρέφει τη λίστα με τα ερωτήματα της συλλογής.

            Στις συναρτήσεις {\fontCode\small \textbf{preprocess\_collection}()} και {\fontCode\small \textbf{preprocess\_queries}()}
            πραγματοποιείται η προεπεξεργασία των εγγράφων, συγκεκριμένη η αφαίρεση των \textbf{stopwords} και το \textbf{stemming}.


            Η αφαίρεση των stopwords και το stemming γίνεται με τη χρήση της {\fontCode\small nltk} βιβλιοθήκης. Τα {\fontCode\small doc\_tuples} της
            {\fontCode\small get\_docs()} αφού περάσουν από τον {\fontCode\small PorterStemmer} της {\fontCode\small nltk} αποθηκεύονται σε μια λίστα,
            η οποία στη συνέχεια επιστρέφεται.
            Αντίστοιχη διαδικασία πραγματοποιείται και για την προεπεξεργασία των ερωτημάτων, στην {\fontCode\small preprocess\_queries()}.

        \section{ΑΝΕΣΤΡΑΜΜΕΝΟ ΕΥΡΕΤΗΡΙΟ}

            Το ανεστραμμένο ευρετήριο δημιουργείται στη συνάρτηση {\fontCode\small \textbf{create\_inverted\_index}()} του αρχείου {\fontCode\small inverted\_index.py}.
            Στην συνάρτηση εισαγάγονται οι λίστες που δημιουργήθηκαν στις προηγούμενες συναρτήσεις.

            Τα tuples που αντιστοιχούν σε αυτά αποθηκεύονται σε ένα dictionary που θα αποτελέσει το ανεστραμμένο ευρετήριο με την εξής δομή:

                \begin{graycomment} \centering
                    {\fontCode\footnotesize inverted\_index['λήμμα'] = \\ \{ ('docID στο οποίο εμφανίζεται' = <φορές εμφάνισης>), (\(\cdots\)), \(\ldots\) \}}
                \end{graycomment}

            Κάθε value του dictionary είναι ένα set\footnote{Έχει επιλεχθεί set για εξοικονόμιση μνήμης, μας και δεν μας ενδιαφέρει η σειρά των tuples.} το οποίο περιλαμβάνει ένα ή περισσότερα tuples
            με το {\fontCode\small docID} και τη συχνότητα εμφάνισης του λήμματος στο συγκεκριμένο έγγραφο.
            Η συχνότητα υπολογίζεται μέσω της {\fontCode\small count()} σε όλο το έγγραφο ανά λήμμα. Αυτό είναι ένα παράδειγμα του τελικού ανεστραμμένου ευρετηρίου \footnote{Τα λήμματα έχουν τη stemming μορφή τους.}:


                \begin{graycomment} \centering
                    {\fontCode\scriptsize inverted\_index = \{\(\ldots\) 'coronari': {('01217', 2), ('00779', 1), ('00164', 1)}, \\ 'graft': {('00164', 1)}, 'mobil': {('00673', 2), 'strain': {('00179', 7), \(\ldots\)\} }
                \end{graycomment}

        \section{ΥΛΟΠΟΙΗΣΗ VECTOR SPACE ΜΟΝΤΕΛΟΥ}

            Η υλοποίηση του Vector Space μοντέλου πραγματοποιείται στο αρχείο {\fontCode\small vsm.py}, στη συνάρτηση {\fontCode\small \textbf{run\_vsm}()},
            η οποία εναλλάσσει τα ερωτήματα, τα οποία στη συνέχεια εισάγονται στην {\fontCode\small \textbf{vsm}()}, όπου υπολογίζεται όντως η ομοιότητα του συνημιτόνου των διανυσμάτων μεταξύ εγγράφου και ερωτήματος.

            Στόχος είναι να υπολογίσουμε τις tf τιμές από κάθε λήμμα του εκάστοτε ερωτήματος και στη συνέχεια, αν είναι εφικτό, να τα αντιστοιχήσουμε με τα λήμματα των εγγράφων χρησιμοποιώντας το ανεστραμμένο ευρετήριο που έχουμε δημιουργήσει.
            Συνεπώς αφού υπολογίσουμε ...


\end{document}